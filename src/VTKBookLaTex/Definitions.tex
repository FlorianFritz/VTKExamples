
%%% ----------------------------------------------------------------------

% Packages
\usepackage[english]{babel}
\usepackage{csquotes}
%\usepackage[nonumberlist]{glossaries}
\usepackage{glossaries}
\usepackage[T1]{fontenc}
\usepackage{amsmath}
\usepackage{mathtools}
\usepackage{amsfonts}
\usepackage{amssymb}
%\usepackage{mathrsfs}
\usepackage{bigints}
\usepackage{relsize}


\usepackage{tocloft}
\newcommand{\listequationsname}{List of Equations}
\newlistof{myequations}{equ}{\listequationsname}
\newcommand{\myequations}[1]{%
\addcontentsline{equ}{myequations}{\protect\numberline{\theequation}#1}\par}
\setlength{\cftmyequationsnumwidth}{2.5em}% Width of equation number in List of Equations

% For decimal aligned columns.
\usepackage{dcolumn}
\newcolumntype{d}[1]{D{.}{.}{#1}}
% Keeps figures in the sections where they are defined.
\usepackage[section]{placeins}
\usepackage{txfonts}
\usepackage{graphicx}
%\usepackage{grfext}
%\PrependGraphicsExtensions*{.png,.PNG}
% [hypcap=false] gets rid of this message:
% The option `hypcap=true' will be ignored for this(caption) particular \caption
% when \captionof is used.
\usepackage[hypcap=false]{caption}
\usepackage{floatrow}
\usepackage{wrapfig}
\usepackage{floatrow}
\usepackage{subcaption}
\usepackage{enumitem}
\usepackage[dvipsnames]{xcolor}
\usepackage{array}
\usepackage{lettrine}
\usepackage{indentfirst}
\usepackage{titlesec}
\titleformat{\chapter}{\large\bfseries}{\thesection}{1em}{\hrule}
\usepackage{listings}
% Set the default code style.
\lstset{
basicstyle=\normalsize\ttfamily\color{Sepia},
    frame=tb, % draw a frame at the top and bottom of the code block
    tabsize=2, % tab space width
    showstringspaces=false, % don't mark spaces in strings
    numbers=left, % display line numbers on the left
    commentstyle=\color{Green}, % comment color
    keywordstyle=\color{Blue}, % keyword color
    stringstyle=\color{Maroon} % string color
}

\usepackage[
    refsection=chapter,
    backend=biber,
%    bibstyle=numeric,
]{biblatex}
\DeclareFieldFormat{labelalpha}{\thefield{entrykey}}
\DeclareFieldFormat{extraalpha}{}

\addbibresource{Bibliography.bib}

% Make an index
\usepackage{makeidx}

\usepackage{url}
%% Load the hyperref package last.
\usepackage{hyperref}
\hypersetup{
    colorlinks = true,
    linkbordercolor = {White},
    urlcolor = {Blue},
    linkcolor = {Brown},
    citecolor = {BlueViolet}
}

%%% ----------------------------------------------------------------------
% Some general global declarations

% Graphics paths and extensions.
\graphicspath{{Art/}{Art/Scraped/}{Figures/}{Figures/Scraped/}{Figures/Headings/}}
% Prefer pdf over png.
\DeclareGraphicsExtensions{%
    .pdf,.PDF,%
    .png,.PNG,%
    .jpg,.mps,.jpeg,.jbig2,.jb2,.JPG,.JPEG,.JBIG2,.JB2}

\addtolength{\textwidth}{50pt}
\addtolength{\evensidemargin}{-25pt}
\addtolength{\oddsidemargin}{-25pt}

\def\displayandname#1{\rlap{$\displaystyle\csname #1\endcsname$}%
                      \qquad \texttt{\char92 #1}}
\def\mathlexicon#1{$$\vcenter{\halign{\displayandname{##}\hfil&&\qquad
                   \displayandname{##}\hfil\cr #1}}$$}

\newtheorem{theorem}{Theorem}[section]
\newtheorem{lemma}[theorem]{Lemma}
\newtheorem{proposition}[theorem]{Proposition}
\newtheorem{corollary}[theorem]{Corollary}

\newenvironment{proof}[1][Proof]{\begin{trivlist}
\item[\hskip \labelsep {\bfseries #1}]}{\end{trivlist}}
\newenvironment{definition}[1][Definition]{\begin{trivlist}
\item[\hskip \labelsep {\bfseries #1}]}{\end{trivlist}}
\newenvironment{example}[1][Example]{\begin{trivlist}
\item[\hskip \labelsep {\bfseries #1}]}{\end{trivlist}}
\newenvironment{remark}[1][Remark]{\begin{trivlist}
\item[\hskip \labelsep {\bfseries #1}]}{\end{trivlist}}

\newcommand{\qed}{\nobreak \ifvmode \relax \else
      \ifdim\lastskip<1.5em \hskip-\lastskip
      \hskip1.5em plus0em minus0.5em \fi \nobreak
      \vrule height0.75em width0.5em depth0.25em\fi}



%\newcommand {\myfrac } [2] { \frac{\displaystyle #1}{\displaystyle #2} }
\newcommand {\half} {\dfrac{1}{2}}
\newcommand {\onehalf} {\dfrac{3}{2}}
\newcommand{\firstletter}[1] {\lettrine[lines=2, lraise=0.7]{\color{Sepia}\textit{#1}}}

%\newcommand{\listcpp}[1] {{\lstlisting[style=mycpp]{#1}}}
%%% ----------------------------------------------------------------------

  % Horizontal spacing.
  % \, small space 3/18 quad.
  % \: medium space 4/18 quad.
  % \; large space 5/18 quad.
  % \! negative space -3/18 quad.
%%% ----------------------------------------------------------------------
