\chapter{Advanced Computer Graphics}
\label{chap:acg}

\firstletter{C}hapter 3 introduced fundamental concepts of computer graphics.
A major topic in that chapter was how to represent and render geometry using surface primitives such as points, lines, and polygons.
In this chapter our primary focus is on volume graphics.
Compared to surface graphics, volume graphics has a greater expressive range in its ability to render inhomogeneous materials, and is a dominant technique for visualizing 3D image (volume) datasets.

We begin the chapter by describing two techniques that are important to both surface and volume graphics.
These are simulating object transparency using simple blending functions, and using texture maps to add realism without excessive computational cost.
We also describe various problems and challenges inherent to these techniques.
We then follow with a focused discussion on volume graphics, including both object-order and image-order techniques, illumination models, approaches to mixing surface and volume graphics, and methods to improve performance.
Finally, the chapter concludes with an assortment of important techniques for creating more realistic visualizations.
These techniques include stereo viewing, antialiasing, and advanced camera techniques such as motion blur, focal blur, and camera motion.

\section{3D Widgets and User Interaction}
\label{sec:3dwui}
Chapter 3 provided an introduction to interaction techniques for graphics (see ``Introducing vtkRenderWindowInteractor'' on page \pageref{pg:rwi} ).
In the context of visualization, interaction is an essential featureof systems that provide methods for data exploration and query. The classes
vtkRenderWindowInteractor and vtkInteractorStyle are core constructs used in VTK to capturewindowing-system specific events in the render window, translate them into VTK events, and then take action as appropriate to that event invocation.
In Chapter 3 we saw how these classes could be used to manipulate the camera and actors to interactively produce a desired view. This functionality,
however, is relatively limited in its ability to interact with data. For example, users often wish tointeractively control the positioning of streamline starting points, control the orientation of a clipping plane, or transform an actor. While using interpreted languages (see ``Interpreted Code'' on page \pageref{pg:rwi} ) can go a long way to provide this interaction, in some situations the ability to see what you are doing when placing objects is essential. Therefore, it is apparent that a variety of user interaction techniques is required by the visualization system if it is to successfully support real-world applications.
\section{Transparency and Alpha Values}