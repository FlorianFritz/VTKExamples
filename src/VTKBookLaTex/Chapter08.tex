\chapter{Advanced Data Representation}
\label{chap:adr}

\firstletter{T}his chapter examines advanced topics in data representation.
Topics include topological and geometric relationships and computational methods for cells and datasets.

\section{Coordinate Systems}
We will examine three different coordinate systems: the global, dataset, and structured coordinate systems.
Figure8–1 shows the relationship between the global and dataset coordinate systems, and depicts the structured coordinate system.

\subsection{Global Coordinate System}
The global coordinate system is a Cartesian, three-dimensional space. Each point is expressed as a triplet of values $(x,y,z)$ along the $x$, $y$, and $z$ axes.
This is the same system that was described in Chapter 3 (see \ref{sec:coordinate_systems}).
The global coordinate system is always used to specify dataset geometry (i.e., the point coordinates),and data attributes such as normals and vectors.
We will use the word ``position'' to indicate that we are using global coordinates.

\section{Putting It All Together}
In this section we will finish our earlier description of an implementation for unstructured data. We also define a high-level, abstract interface for cells and datasets. This interface allows us to implement the general (i.e., dataset specific) algorithms in the \emph{Visualization Toolkit}. We also describe
implementations for color scalars, searching and picking, and conclude with a series of examples to demonstrate some of these concepts.

\subsection{Picking}
\label{sec:picking}

The Visualization Toolkit provides a variety of classes to perform actor (or vtkProp), point, cell, and
world point picking ( Figure8–38 ).